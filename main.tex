\documentclass[aspectratio=169,professionalfonts, 12pt, t]{beamer}
\usepackage{lmodern}

\input{misc/variables}
\title[\shortTitle]{\presentationTitle{}}
\subtitle{\presentationSubtitle}
\author[\presentationAuthorShort]{\large \presentationAuthor}
 
\date{\today} 

%
% XeTeX specific
%
%\RequireXeTeX
 \usepackage{xltxtra}
%\usepackage{fontspec}
%\usepackage{xunicode}

% -----------------------------------
% font config
% -----------------------------------

%\setsansfont{Linux Biolinum}
%\setromanfont{Linux Libertine}
%\setmonofont[Scale=0.9]{Consolas}

% -----------------------------------
% math
% -----------------------------------
\usepackage{lmodern}

\usepackage[utf8]{inputenc} % this is needed for german umlauts
\usepackage[frenchb]{babel} % this is needed for german umlauts
\usepackage[T1]{fontenc}    % this is needed for correct output of umlauts in pdf
\usepackage[francais,nohints]{minitoc}		% Mini table des matières, en français

\usepackage{amsmath}
\usepackage{amsfonts}
\usepackage{amssymb}
\usepackage{amsthm}
\usepackage{mathrsfs}   % \mathscr
\usepackage{stmaryrd}   % \lightning

% -----------------------------------
% grammar and textstyle
% -----------------------------------

\usepackage{polyglossia}
\setdefaultlanguage{french}
\usepackage{csquotes}
% underlining
\usepackage{soul} % ulem redifines \emph, this sucks

% -----------------------------------
% colors
% -----------------------------------

\usepackage{xcolor}
\usepackage{colortbl}

% listing colors
% https://kuler.adobe.com/Color-palette-color-theme-4004722
\definecolor{keyword}{RGB}{239,33,74}
\definecolor{number}{RGB}{243,202,22}
\definecolor{comment}{RGB}{126,158,19}
\definecolor{string}{RGB}{6,129,128}
\usepackage{fontawesome}
 

\usepackage{tikz}
\usetikzlibrary{positioning, arrows}
\usepackage{animate}
\usepackage{graphicx}
% -----------------------------------
% links and references
% -----------------------------------

\usepackage{hyperref}
\hypersetup{
    colorlinks=false,
    % hide bookmarks
    pdfpagemode=UseNone,
    % meta
    pdfauthor={\presentationAuthor},
    pdftitle={\presentationTitle}
}
\usepackage[german]{cleveref}

% -----------------------------------
% bibliography and glossaries
% -----------------------------------
 

 %\usepackage[
 %    backend=biber,
  %   style=alphabetic-verb
  %  ]{biblatex}

\usepackage[]{natbib}
%\usepackage{bibentry}
\renewcommand{\bibsection}{\section*{Bibliographie}}
\setcitestyle{authoryear,open={[},close={]}}
%\usepackage{glossaries}
%\input{glossary}
%\makeglossaries

% -----------------------------------
% graphics
% -----------------------------------

\usepackage{graphicx}
\usepackage{tikz}
\usetikzlibrary{shapes.geometric, arrows, shadows, positioning,matrix}
\usepackage{pgffor}
\usepackage{adforn} % ornaments, used in titlepage
\usepackage{multicol}
\tikzset{
	invisible/.style={opacity=0},
	visible on/.style={alt={#1{}{invisible}}},
	alt/.code args={<#1>#2#3}{%
		\alt<#1>{\pgfkeysalso{#2}}{\pgfkeysalso{#3}} % \pgfkeysalso doesn't change the path
	},
}
% -----------------------------------
% beamer theme
% -----------------------------------

% you can't locate the theme in a subfolder without shooting yourself in the knee
\usetheme{alinz}

% -----------------------------------
% listings and pseudocode
% -----------------------------------

\Crefname{lstlisting}{Listing}{Listings}
\crefname{lstlisting}{listing}{listings}

\usepackage{listings}
\lstset{
    basicstyle=\footnotesize\ttfamily\color{lightdark},
    backgroundcolor=\color{blockbg},
    numbers=left,
    %numbersep=6pt,
    numberstyle=\scriptsize\color{granite},
    commentstyle=\sffamily\itshape\color{sea},
    keywordstyle=\bfseries\color{raspberry},
    stringstyle=\itshape\color{lake},
    showstringspaces=false,
    breaklines=true,
    breakatwhitespace=true,
    frame=lr,
    framerule=0pt,
    framesep=6pt,
    captionpos=b
}
% for pseudocode
\usepackage[slide,linesnumbered,algoruled]{algorithm2e}
\usepackage{pst-plot}
\usepackage{pst-solides3d}
\xdefinecolor{pgold}{rgb}{0.8,0.6,0.2}

\psset{viewpoint=50 20 10 rtp2xyz,Decran=50,linewidth=0.5\pslinewidth}
% These next six lines declare a new beamerboxes environment
%\setbeamercolor{uppercol}{ffg=blocktitlefg, bg=titlebg}
\setbeamercolor{lowercol}{fg=lightdark, bg=blockbg}
\newenvironment{colorblock}
{
	\begin{beamerboxesrounded}[lower=lowercol,shadow=true]}
	{\end{beamerboxesrounded}}

\usetikzlibrary{arrows,shapes}
\usepackage{verbatim}
% Define some styles for graphs
\tikzstyle{vertex}=[circle,fill=black!25,minimum size=20pt,inner sep=0pt]
\tikzstyle{selected vertex} = [vertex, fill=red!24]
\tikzstyle{blue vertex} = [vertex, fill=blue!24]
\tikzstyle{edge} = [draw,thick,-]
\tikzstyle{weight} = [font=\small]
\tikzstyle{selected edge} = [draw,line width=5pt,-,red!50]
\tikzstyle{ignored edge} = [draw,line width=5pt,-,black!20]

%\setcitestyle{authoryear,open={[},close={]}}
\input{misc/javascript}
\input{misc/html5}
\usepackage{wrapfig}

%\usepackage[utf8]{inputenc} % this is needed for german umlauts
%\usepackage[ngerman]{babel} % this is needed for german umlauts
%\usepackage[T1]{fontenc}    % this is needed for correct output of umlauts in pdf

\usepackage{verbatim}
\usepackage{tikz}
\usetikzlibrary{arrows,shapes}

% see http://deic.uab.es/~iblanes/beamer_gallery/index_by_theme.html
%\usetheme{Frankfurt}
\usetheme{alinz}
%\usepackage[frenchb]{babel}
%\usepackage[T1]{fontenc}
%\usepackage[utf8]{inputenc}

% \usetheme{default} -> A utiliser dans un premier temps
% \usetheme{Warsaw} -> A utiliser dans un second temps
% position the logo
 \jurya{Pr. Djamel Eddine SAIDOUNI}{Directeur de mémoire}{}
 \juryb{Dr. Bouneb Zine El Abidine}{Co-encadreur}{}


%\title[Faire une présentation en LaTeX avec Beamer]{\LARGE Développement d'une approche de distribution des espaces d'états basé sur la théorie de jeux : Application au model checking distribué}
%\author{Spader}
%\institute{www.siteduzero.com}
%\date{\today}
\begin{document}
	\begin{frame}[plain, t,noframenumbering]
	\titlepage
	\end{frame}
\begin{frame}[plain,noframenumbering]
\frametitle{SOMMAIRE}
\tableofcontents
\end{frame}

		\section{Introduction}

\subsection{Contexte}
\begin{frame}{Contexte}
Ces dernières années plusieurs catastrophes sont dues à des erreurs de spécifications des systèmes développés.
\begin{columns}
	\begin{column}{0.3\textwidth}
		\begin{figure}
			\begin{tikzpicture}		
				\only<1->
			{
				\node [inner sep=-10pt]
				{
					\includegraphics[height=1.2in,width=\columnwidth,trim={0 0 0 0},clip]{resources/Ariane_5}
				};              
			}
			\end{tikzpicture}
			\onslide<2->
			{  
				\caption{Ariane 5}
			}
		\end{figure}
	\end{column}

	\begin{column}{0.3\textwidth}
	\begin{figure}
		\begin{tikzpicture}		
		\only<3->
		{
			\node [inner sep=-10pt]
			{
				\includegraphics[height=1.2in,width=\columnwidth,trim={0 0 0 0},clip]{resources/Patriot}
			};              
		}
		\end{tikzpicture}
		\onslide<3->
		{  
			\caption{Missile Patriote}
		}
	\end{figure}
\end{column}

	\begin{column}{0.3\textwidth}
		\begin{figure}
			\begin{tikzpicture}
			\only<4->
			{
				\node [inner sep=-10pt]
				{
					\includegraphics[height=1.2in,width=\columnwidth,trim={0 0 0 0},clip]{resources/bug2000}
				};
			}
			\end{tikzpicture} 
			\onslide<4->
			{
				\caption{Bug 2000}
			}
		\end{figure}
	\end{column}            
\end{columns}

\uncover<5->{%
La fiabilité de tout système est envisageable, en particulier celle de systèmes critiques.
}
%Vue l'importance de ces systèmes a complexité de ces systèmes il est pratiquement impossible de s'assurer que le système est fiable. 
\end{frame}

%La necessite d'utiliser les methodes formelle 
%les methodes formelle permet de prendre un systeme reel et la specifier formellement
\begin{frame}{}
\centering
\vspace{2.2cm}       
	\Huge 
		\textbf{Comment faire?}
%sachant que les jeux de test ne permet pas d'aboutir à une preuve contrainte
\end{frame}

\begin{frame}{titre section}
%le model checking permet de détecter automatiquement des erreurs dans le processus de
%conception, elle fournit aussi un contre-exemple en cas de non insatiabilité de la propriété dans le modèle permettant ainsi de corriger la source de l’erreur dans le
%système.

\begin{columns}[onlytextwidth,t]
	\column{.3\textwidth}
	\only<1->
	{
	\includegraphics[height=1.2in,width=\columnwidth,clip=true,trim={-2 0 0 0}]{resources/mc/0001}
	}
	\column{.3\textwidth}
	\only<2->
	{
	\includegraphics[height=1.2in,width=\columnwidth,clip=true,trim={0 0 0 0}]{resources/mc/0002}
	}
	\column{.3\textwidth}
	\only<3->
	{
	\includegraphics[height=1.2in,width=\columnwidth,clip=true,trim={0 0 0 0}]{resources/mc/0003}
	}
\end{columns}
\begin{columns}[onlytextwidth,t]
	\column{.3\textwidth}
	\only<4->
	{
		\includegraphics[height=1.2in,width=\columnwidth,clip=true,trim={0 0 0 0}]{resources/mc/0004}
	}
	\column{.3\textwidth}
	\only<5->
	{
		\includegraphics[height=1.2in,width=\columnwidth,clip=true,trim={0 0 0 0}]{resources/mc/0005}
	}
	\column{.3\textwidth}
	\only<6->
	{
		\includegraphics[height=1.2in,width=\columnwidth,clip=true,trim={0 0 0 0}]{resources/mc/0006}
	}
\end{columns}

\end{frame}

\begin{frame}
\centering
\vspace{2.2cm}       
\Huge 
\textbf{Problèmes}
%sachant que les jeux de test ne permet pas d'aboutir à une preuve contrainte
\end{frame}
\subsection{Problèmes}

\begin{frame}{titre section}

	\begin{center}
	 \begin{columns}
	 	\begin{column}{\textwidth}
	 		\begin{figure}
					% Define block styles
				\tikzstyle{decision} = [diamond, draw, fill=blue!20, 
				text width=4.5em, text badly centered, node distance=5cm, inner sep=0pt]
				\tikzstyle{block} = [rectangle, draw, fill=white, 
				text width=5em, text centered, rounded corners, minimum height=4em]
				\tikzstyle{line} = [draw, -latex']
				\tikzstyle{cloud} = [draw, ellipse,red!80, node distance=5cm,
				minimum height=2em]
				
				\begin{tikzpicture}[scale=3, every node/.style={scale=0.4},node distance = 5cm, auto]
				\only<2->
				{
					\node [decision ] (decide) {Peut-il être stocké sur une machine ?};
				}
				
				\node [cloud,fill=white,left of=decide] (g) {\includegraphics[width=1.5cm]{petri}};
				
				\only<3->
				{
					\node [block, below of=decide] (stockable) {\includegraphics[width=1.5cm]{pc}};
				}
				\only<7->
				{
					\node [block, right of=decide,text width=10em, blue,node distance=10cm] (distribution)  {\includegraphics[width=3cm]{graphD}};
				}
				\only<5->
				{	
					\node [decision,below of=stockable] (tempsR) {Temps de reponse rainsonnable ?};
				}
				\only<6->
				{
					\node [block,left of=tempsR,blue ] (m) {\includegraphics[width=2cm]{petri}};
				}
				\only<9->			
				{
					\node [block, right of=distribution,text width=10em,node distance=7cm ] (cdistribuer) {Comment Distribué ?}; 
				}
				% Draw edges
				\path [line,visible on=<2-> ] (g) -- (decide);
				\path [line,visible on=<3-> ] (decide) --node {OUI} (stockable);
				\path [line,visible on=<8-> ] (decide) -- node {NON}(distribution);
				\path [line,visible on=<5-> ] (stockable) --(tempsR);
				\path [line,visible on=<6-> ] (tempsR) -- node {OUI}(m);
				\path [line,visible on=<7-> ] (tempsR) -| node [near start] {NON}(distribution);
				\path [line,visible on=<9-> ] (distribution) -- (cdistribuer);
				
				\end{tikzpicture}

	 		\end{figure}
	 	\end{column}
 	\end{columns}
	\end{center}

\end{frame}


\begin{frame}
\centering
\vspace{2.2cm}       
\Huge 
\textbf{Comment distribué ?}
%chercher a surcharger une machine faira qu'augmenter le temps de la verification
\end{frame}

\subsection{Motivation}
\begin{frame}{Motivation}
\begin{columns}
	\begin{column}{0.4\textwidth}
		\begin{figure}	
			\only<1->
			{
			\begin{tikzpicture}[cap=round,line width=3pt]
		%	\filldraw [fill=examplefill] (0,0) circle (2cm);
			\foreach \angle / \label in
			{0/3, 30/2, 60/1, 90/12, 120/11, 150/10, 180/9,
				210/8, 240/7, 270/6, 300/5, 330/4}
			{
				\draw[line width=1pt] (\angle:1.8cm) -- (\angle:2cm);
				\draw (\angle:1.4cm) node{\textsf{\label}};
			}
			\foreach \angle in {0,90,180,270}
			\draw[line width=2pt] (\angle:1.6cm) -- (\angle:2cm);
			\draw (0,0) -- (120:0.8cm); % hour
			\draw (0,0) -- (90:1cm); % minute
			\end{tikzpicture}%         
			}
		
		\end{figure}
	\end{column}
	
	\begin{column}{0.4\textwidth}
		\begin{figure}
			\begin{tikzpicture}		
			\only<3->
			{
				\node [inner sep=-10pt]
				{
					\includegraphics[width=\columnwidth,trim={0 0 0 0},clip]{resources/bndistributions}
				};              
			}
			\end{tikzpicture}			
		\end{figure}
	\end{column}   
\end{columns}


\end{frame}

	\section{Solutions Proposées}
%les solutions proposées dans la literature sont faites en amont, nous pouvons clasifier ces solutions en 2 categories
\subsection{Première Catégorie}
\begin{frame}{title sub section}
\begin{wrapfigure}{r}{0.3\textwidth}
	\vspace{-35pt}
	\begin{center}
		\includegraphics[width=0.2\textwidth]{resources/partitions}
	\end{center}
	\vspace{-35pt}
	\end{wrapfigure}
       Les approches de cette catégorie aboutissent à une meilleur équilibrage de charge entre les différentes machines.
		\begin{columns}
      		\begin{column}{0.3\textwidth}
      			\begin{figure}
      				\centering
      				\includegraphics[width=\linewidth]{resources/transitions}
      			\end{figure}      			
      		\end{column}
      	
	      	\begin{column}{0.7\textwidth}
	      		\begin{block}{Problèmes}
	      			\begin{itemize}
	      				\item Distribution statique.% la distribution reste intacte tout au long du model cheking
	      				\item Nombre de Transitons externes minimum implique -t-il réduction du taux de communication?.% pourai entrainer un nombre elever de communications
	      				
	      				\item La puissance des machines non exploitée.% il peut y avoir des machines qui effetue peut de calcule alors que les autres un nombre elevé de calcul
	      				\item Temps de réponse déraisonnable.
	      			\end{itemize}
	      		\end{block}
	      	\end{column}
      	\end{columns}	    
\end{frame}

\subsection{Deuxième Catégorie}
\begin{frame}{title sub section}
\begin{wrapfigure}{r}{0.3\textwidth}
	\vspace{-20pt}
	\includegraphics[width=0.3\textwidth]{resources/transition}
	\vspace{-20pt}
\end{wrapfigure}
La philosophie de cette catégorie vise à minimiser les transitions externes avec un bon équilibrage de charge entre les différentes machines.
 
		%\item Distribution statique.% la distribution reste intacte tout au long du model cheking
		\begin{columns}
			\begin{column}{0.3\textwidth}
				\centering
				\begin{figure}
					\begin{tikzpicture}		

						\node [inner sep=-10pt,visible on=<1-1>]{
							\includegraphics[height=1.2in,width=\columnwidth,trim={0 0 0 0},clip]{resources/bequilibre}
						};              
					
					\node [inner sep=-10pt,visible on=<2->]
					{
						\includegraphics[height=1.2in,width=\columnwidth,trim={0 0 0 0},clip]{resources/contre_exemple}
					}; 
					\node[inner sep=-10pt,visible on=<3->](expressions) at (-1.3,-1.8) {AG\(f\)};             
			
					\end{tikzpicture}
				\end{figure}
			\end{column}
		
         \begin{column}{0.7\textwidth}
         	
         	\begin{block}{Problèmes}
         		\begin{itemize}
         		\item Distribution statique.% la distribution reste intacte tout au long du model cheking
         		\item L'équilibrage peut être dégradé.% pourai entrainer un nombre elever de communications
         	%
         		\item La puissance des machines non exploitée.% il peut y avoir des machines qui effetue peut de calcule alors que les autres un nombre elevé de calcul         		
         	 \end{itemize}
           \end{block}
         \end{column}
		\end{columns}
 \vspace{0.5cm}
\textbf{ Minimisation des transitions externes \color{red} $\stackrel{?}{\Rightarrow}$ Temps de réponse minimisé}.  
\end{frame}

\subsection{Solution en aval}
\begin{frame}
	\centering
	\vspace{2.2cm}       
	\Huge 
	\textbf{Solution en aval}
	%chercher a surcharger une machine faira qu'augmenter le temps de la verification
\end{frame}

\begin{frame}{title sub section}
  
  
  
\begin{center}
	\begin{columns}
		\begin{column}{\textwidth}
			\begin{figure}				
				\begin{tikzpicture}		
				\only<1-1>
				{
					\node [inner sep=-10pt]
					{
						\includegraphics[height=1.8in,width=0.7\columnwidth,trim={0 0 0 0},clip]{resources/benstira/0001}
					};              
				}
			\only<2-2>
			{
				\node [inner sep=-10pt]
				{
					\includegraphics[height=1.8in,width=0.7\columnwidth,trim={0 0 0 0},clip]{resources/benstira/0002}
				};              
			}
			\only<3-3>
			{
				\node [inner sep=-10pt]
				{
					\includegraphics[height=2in,width=0.7\columnwidth,trim={0 0 0 0},clip]{resources/benstira/0003}
				};              
			}
			\only<4-4>
			{
				\node [inner sep=-10pt]
				{
					\includegraphics[height=2in,width=0.7\columnwidth,trim={0 0 0 0},clip]{resources/benstira/0004}
				};              
			}		
				\end{tikzpicture}				
				
			\end{figure}
		\end{column}
	\end{columns}
\end{center}
\end{frame}

\begin{frame}
	\centering
	\vspace{2.2cm}       
	\Huge 
	\textbf{Critiques}	
\end{frame}

\begin{frame}{tite subsection}
\vspace{-10pt}
\begin{block}{Critiques}
	\vspace{-15pt}
	\begin{columns}
		\begin{column}{\textwidth}
			\begin{itemize}
				\item Minimisations des transitions externes.
				\item Duplications et Migrations basées sur les transitions.
				\item Certains machines peuvent être surchargées de calcul ou de stockage.
				\item Duplication de certains états est sans intérêt.% sinon le model checking sera faux
			\end{itemize}
		\end{column}
	\end{columns}

	\begin{columns}
		\begin{column}{0.3\textwidth}
			\centering
			\begin{figure}				
				\begin{tikzpicture}		
					\only<1-1>
					{
						\node [inner sep=-10pt]
						{
							\includegraphics[height=1.2in,width=\columnwidth,trim={0 0 0 0},clip]{resources/benstira/critque1}
						};              
					}
					
					\node [inner sep=-10pt,visible on=<2-2>]
					{
						\includegraphics[height=1.2in,width=\columnwidth,trim={0 0 0 0},clip]{resources/benstira/critque2}
					};  
				
					\node [inner sep=-10pt,visible on=<3-3>]
					{
						\includegraphics[height=1.2in,width=\columnwidth,trim={0 0 0 0},clip]{resources/benstira/critque3}
					};                
			% question est ce que le temps est amelioré? si oui le model checking sera fausse, sinon la duplication n'a rien servi
				\end{tikzpicture}				
			\end{figure}
		\end{column}
		\begin{column}{0.7\textwidth}
			\begin{itemize}
				\item AG(a)
				\item Si 0.45 < L/$N_t$ ≤ 0.75, alors dupliqué \color{red}{\tiny \citep{BENSETIRA2017}}.
			\end{itemize}
		\end{column}
	\end{columns}
\end{block}
\end{frame}

\subsection{Partitionnent}
\begin{frame}{tite subsection}
\vspace{-10pt}
\begin{block}{Critiques}
	\vspace{-15pt}
	\begin{columns}
		\begin{column}{\textwidth}
			\begin{itemize}
				\item Minimisations des transitions externes.
				\item Duplications et Migrations basées sur les transitions.
				\item Certains machines peuvent être surchargées de calcul ou de stockage.
				\item Duplication de certains états est sans intérêt.% sinon le model checking sera faux
			\end{itemize}
		\end{column}
	\end{columns}
	
\end{block}
\end{frame}

	\section{Contribution}

% nous allons trouver un compromis entre le temps d'execution et l'equilibrage de charge
%\begin{frame}
%\centering
%	\vspace{2.2cm}       
%	\Huge 
%	\textbf{Pourquoi \\la réduction des transitions externes \\et\\ l'équilibrage des états \\n'aide pas le model checking ?}	
%\end{frame}

% avant tout, presentons les point de partitions
%tout partition outre les point que la formule n'est pas verifier initialement entrainera un nombre de calcule. les exemples precedents le demontre.

%il peut arriver que les machine ne peuvent pas supporter ces etats par leur nombres
%le mieux est de le surcharger en calcule ou en stockage que les deux en meme temps
\subsection{Points de partitions}
\begin{frame}{Définition}
  \centering
   \begin{columns}
   	\begin{column}{0.8\textwidth}
   		\vspace{-20pt}
   		\begin{figure}				
   			\begin{tikzpicture}		
   			\only<1-1>
   			{
   				\node [inner sep=-10pt]
   				{
   					\includegraphics[height=1.2in,width=\columnwidth,trim={0 0 0 0},clip]{resources/benstira/critque1}
   				};              
   			}
   			
   			\node [inner sep=-10pt,visible on=<2->]
   			{
   				\includegraphics[height=1.2in,width=\columnwidth,trim={0 0 0 0},clip]{resources/pointpartition}
   			};
   		
   			\end{tikzpicture}				
   		\end{figure}
   	\end{column}   	
   \end{columns}

\begin{columns}
	\begin{column}{0.8\textwidth}
		\vspace{15pt}
		\begin{figure}				
			\begin{tikzpicture}	
			\node [inner sep=-10pt,visible on=<3->]
			{
				\includegraphics[height=1in,width=0.5\columnwidth,trim={0 0 0 0},clip]{resources/pc}
			};			
			\end{tikzpicture}				
		\end{figure}
	\end{column}   	
\end{columns}
\end{frame}

\subsection{Équilibre de Nash}
\begin{frame}{Définition}
\centering
\vspace{-20pt}
\begin{columns}
	\begin{column}{0.3\textwidth}
		\begin{figure}
			\centering
			\includegraphics[width=\linewidth]{resources/nash}
		\end{figure}
		
	\end{column}
	\begin{column}{0.7\textwidth}
		\begin{itemize}
			\item Une situation o\`{u} adopte la meilleure réponse du choix des autres.
			\item Il lui a valu le  \textbf{Prix Nobel}  d'économie en 1994.
			%L'existence d'un équilibre n'implique pas que celui-ci soit nécessairement optimal
		\end{itemize}
	\centering
	\includegraphics[width=0.5\linewidth]{resources/prisonier}
	\end{column}   	
\end{columns}
\end{frame}


%nous comptons utiliser cette philosophie pour equilibrer le temps de la verification tout en degradant le stockage

\subsection{Stratégie de Distribution}
\begin{frame}{title subsection}
\centering
\vspace{-20pt}
\begin{columns}
	\begin{column}{1\textwidth}
		\begin{figure}
			\centering
			\begin{tikzpicture}	
			\node [inner sep=-10pt,visible on=<1-1>]
			{
				\includegraphics[width=\columnwidth,trim={0 0 0 0},clip]{resources/d1}
			};	
			\node [inner sep=-10pt,visible on=<2-2>]
			{
				\includegraphics[width=\columnwidth,trim={0 0 0 0},clip]{resources/d2}
			};			
			 
			 \node [inner sep=-10pt,visible on=<3-3>]
			 {
			 	\includegraphics[width=\columnwidth,trim={0 0 0 0},clip]{resources/d3}
			 };	
		 
		 	\node [inner sep=-10pt,visible on=<4-4>]
		 	{
		 		\includegraphics[width=\columnwidth,trim={0 0 0 0},clip]{resources/d5}
		 	};	
	 	
	 	    \node [inner sep=-10pt,visible on=<5-5>]
	 	    {
	 	    	\includegraphics[width=\columnwidth,trim={0 0 0 0},clip]{resources/d4}
	 	    };	
 	    
 	    	\node [inner sep=-10pt,visible on=<6-6>]
 	    	{
 	    		\includegraphics[width=\columnwidth,trim={0 0 0 0},clip]{resources/d6}
 	    	};	
     	
     		\node [inner sep=-10pt,visible on=<7-7>]
     		{
     			\includegraphics[width=\columnwidth,trim={0 0 0 0},clip]{resources/d7}
     		};	
     	
     		\node [inner sep=-10pt,visible on=<8-8>]
     		{
     			\includegraphics[width=\columnwidth,trim={0 0 0 0},clip]{resources/d8}
     		};	
     	
			\end{tikzpicture}
		\end{figure}		
	\end{column} 
	
\end{columns}
\end{frame}
%nous comptons a realiser une meilleur partion
\subsection{Model checking par déduction}
\begin{frame}{title subsection}
	\begin{block}
	
	\begin{itemize}
		\item Notion de duplicata
		\item Déduit la valeur logique des duplicatas
		\item Minimise le taux de communications
	\end{itemize}
	\end{block}
\end{frame}

%ces deux approche permet d'accelerer le temps de la verification d'une propriete 
	\section{Conclusion}
\subsection{Conclusion}
\begin{frame}{Conclusion}
 
\end{frame}

\subsection{Perspectives}
\begin{frame}{Perspectives}
\begin{block}{Perspectives}
	\begin{itemize}[<+->]
		\transdissolve
		\transduration{2}
		\item   Explorée les différentes stratégies de la théorie de jeux pour apporter des améliorations supplémentaires à notre stratégie afin d’aboutir à une meilleur stratégie de distribution.
		\item  Extraire un modèle de partitionnement basé sur le machine learning à partir des différentes statistiques générées durant l’exécution du model cheking.
	\end{itemize}
 
\end{block}
\end{frame}


\begin{frame}
\begin{itemize}[<+->]
	\transdissolve
	\transduration{2}
	\item Large number of possible parameter-value combinations
	\item Hard to find the optimal parameters
	\item Which parameters should be changed and by how much. 
	\item muliticollinearity or high correlation between parameter values
	\item Which criteria for evaluating the difference between observed and 
	simulated runoff.
\end{itemize}
{\tiny }\end{frame}

%\input{ressource/animation}

\end{document}